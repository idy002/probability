\documentclass[12pt]{article}
%\usepackage{xeCJK}
\usepackage{amsmath}
\usepackage{amssymb}
\usepackage{amsthm}
\providecommand{\abs}[1]{\lvert#1\rvert}
\providecommand{\norm}[1]{\lVert#1\rVert}

\newtheorem{thm}{Theorem}
\newtheorem{lemma}[thm]{Lemma}
\newtheorem{fact}[thm]{Fact}
\newtheorem{cor}[thm]{Corollary}
\newtheorem{eg}{Example}
\newtheorem{ex}{Exercise}
\newtheorem{defi}{Definition}
\theoremstyle{definition}
\newtheorem{hw}{Problem}
\newenvironment{sol}
{\par\vspace{3mm}\noindent{\it Solution}.}
{\qed}

\newcommand{\ov}{\overline}
\newcommand{\cb}{{\cal B}}
\newcommand{\cc}{{\cal C}}
\newcommand{\cd}{{\cal D}}
\newcommand{\ce}{{\cal E}}
\newcommand{\cf}{{\cal F}}
\newcommand{\ch}{{\cal H}}
\newcommand{\cl}{{\cal L}}
\newcommand{\cm}{{\cal M}}
\newcommand{\cp}{{\cal P}}
\newcommand{\cs}{{\cal S}}
\newcommand{\cz}{{\cal Z}}
\newcommand{\eps}{\varepsilon}
\newcommand{\ra}{\rightarrow}
\newcommand{\la}{\leftarrow}
\newcommand{\Ra}{\Rightarrow}
\newcommand{\dist}{\mbox{\rm dist}}
\newcommand{\bn}{{\mathbf N}}
\newcommand{\bz}{{\mathbf Z}}

\setlength{\parindent}{0pt}
%\setlength{\parskip}{2ex}
\newenvironment{proofof}[1]{\bigskip\noindent{\itshape #1. }}{\hfill$\Box$\medskip}

\renewcommand{\familydefault}{pnc}

\begin{document}
	
	\bigskip
	
	\begin{hw}
		There is a game. One starts from $(0,0)$ and walks to $(n,n)$, allowed to walk towards right and walk towards up. Define the number of $k$ upward walks above the straight line from $(0,0)$ to $(n,n)$ to be $f(n,k)$. We have that 
		$$
			f(n,k) = \frac{1}{n+1}\binom{2n}{n} = C_n
		$$
	\end{hw}
	
	\begin{proof}
		Proof by induction. We define the proposition $P(N)$ to be that the property described in the problem holds for all $0 \leq n \leq N$. 
		
		It is easy to check that $P(0)$ and $P(1)$ is true. 
		
		Assuming that $P(N)$ is true, we will prove that $P(N+1)$ is also true.
		
		We enumerate the last position $(x,y)$ the person arrives that satisfies $x = y$. Assume the position is $(r,r)$, where $0 \leq r \leq N$. We use the technique of series to express the schemes:
		$$
			(f(r,0)x^0+f(r,1)x^1+f(r,2)x^2+\cdots +f(r,r)x^r)(C_{N-r}x^0 + C_{N-r}x^{N-r})
		$$
		where the coefficient of $x^i$ equals to the number of schemes that there are $i$ upward walks from $(0,0)$ to $(N+1,N+1)$ above the straight line in such condition. The second part is the schemes that walks from $(r,r)$ to $(N+1,N+1)$ and never arrives the positions on the straight line between $(r,r)$ and $(N+1,N+1)$. 
		
		By the assumption, $f(r,i) = C_r$, this formula can also be expressed as
		$$
			(C_rx^0+C_rx^1+C_rx^2+\cdots +C_rx^r)(C_{N-r}x^0 + C_{N-r}x^{N+1-r})
		$$
		We can sum them up when $0 \leq r \leq N$:
		\[
		\begin{split}
		 & \sum_{r = 0}^{N}(C_rx^0+C_rx^1+C_rx^2+\cdots +C_rx^r)(C_{N-r}x^0 + C_{N-r}x^{N+1-r}) \\ 
		=& \sum_{r = 0}^{N}C_rC_{N-r}(x^0+x^1+\cdots+x^r)(x^0+x^{N+1-r}) \\
		=& \sum_{r = 0}^{N}C_rC_{N-r}(x^0+x^1+\cdots+x^r) + \sum_{r = 0}^{N}C_rC_{N-r}(x^{N+1-r}+\cdots+x^{N+1}) \\
		=& \sum_{r = 0}^{N}C_rC_{N-r}(x^0+x^1+\cdots+x^r) + \sum_{r = 0}^{N}C_rC_{N-r}(x^{r+1}+\cdots+x^{N+1}) \\
		=& \sum_{r = 0}^{N}C_rC_{N-r}(x^0+x^1+\cdots+x^{N+1}) \\
		=& C_{N+1}(x^0+x^1+\cdots+x^{N+1})
		\end{split}
		\],
		which means that $f(N+1,r) = C_{N+1}$ for all $0 \leq r \leq N+1$ and $P(N+1)$ is true.
	\end{proof}
	
	\hfill by dyy

\end{document}
